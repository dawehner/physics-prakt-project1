\section{Das Zwei-Körper Problem [Aufgabe 2]}
Die erste Aufgabe bestand darin ein Zwei Körper Problem zu simulieren.
Dabei waren die Anfangsdaten in der Datei ``in2'' gegeben und lauten folgendermaßen:
\newline\newline
\begin{center}
\begin{tabular}{c|ccc|ccc}
  Masse & x & y & z & $v_x$ & $v_y$ & $v_z$ \\
  \hline
  1.0 & 1.0 & 0 & 0    & 0 & 0 & 0 \\
  1.0 & 0 & 0 & 0    & 0 & -1.0 & 0 \\
\end{tabular}
\end{center}
Das System befindet sich dabei im Koordinatensystems des ersten Körpers.\\
Wie man im Bild Abbildung \ref{fig:2body:rk4:positions} sieht ergeben die Anfangsdaten
eine Kreisbahn.\newline
\begin{figure}
  \centering
  \includegraphics[width=0.8\textwidth]{./images/2/positions.png}
  \label{fig:2body:rk4:positions}
  \caption{Position der zwei Körper mit RK4 und $\eta = 0.001$}
\end{figure}
\newline
\subsection{Qualität der Integratoren}
Um die Qualität der Integratoren zu bestimmten betrachet man die Veränderungen
der Erhaltsungsgrößen über die Zeit bei verschiedenen Schrittweiten $\eta$.
Zur Verfügung steht die Energie $\log\left| E - E_{\text{start}} \right|$,
die Exzentrizität $\log\left|\left|\vec e\right| - \left|\vec{e}_\text{start}\right| \right|$
und die große Halbachse $\log \left| a - e_{\text{start}} \right|$.

\subsubsection{Euler}
Berechnet man mit der Verfahren Euler die Position der beiden Teilchen
und einem $eta = 0.1$ so sieht man in Abbildung \ref{fig:2body:euler:axis}
das die große Halbachsen mit der Zeit ansteigt. Das Verfahren
erzeugt also keine stabilen Bahnen. Zwar verbessert sich dieser Fehler
durch eine Verminderung der Schrittweite, jedoch ergibt sich auch hier
langfristig keine stabile Bahnen, wenn man einen Extra Plot erstellt.
Damit eignet sich Euler, wie zu erwarten, nicht für eine sinnvollen Simulation
des N-Körper Problems.
\begin{figure}
  \begin{center}
    \includegraphics[width=0.8\textwidth]{./images/2/energy-Euler.png}
  \end{center}
  \caption{Relative Energieänderung berechnet mit Euler}
  \label{fig:2body:euler:energy}
\end{figure}
\begin{figure}
  \begin{center}
    \includegraphics[width=0.8\textwidth]{./images/2/axis-Euler.png}
  \end{center}
  \caption{Relative \"Anderung der großen Halbachse berechnet mit Euler}
  \label{fig:2body:euler:axis}
\end{figure}
\begin{figure}
  \begin{center}
    \includegraphics[width=0.8\textwidth]{./images/2/excentric-Euler.png}
  \end{center}
  \caption{Relative \"Anderung der Exzentrizität berechnet mit Euler}
  \label{fig:2body:euler:excentric}
\end{figure}
\begin{figure}
  \begin{center}
    \includegraphics[width=0.8\textwidth]{./images/2/momentum-Euler.png}
  \end{center}
  \caption{Relative \"Anderung des Spezifischen Drehimpuls berechnet mit Euler}
  \label{fig:2body:euler:spec_angular_momentum}
\end{figure}

\subsubsection{Heun/Mittelungs-Methode}
Eine Verbessung der Euler Verfahren ist die Mittelungsmethode.
Betrachet man die Energie in Abbildung \ref{fig:2body:heun:energy}
so sieht man, dass die Energie deutlich weniger zunimmt im Vergleich zu Euler.
Dies bedeutet es sind geschlossene Bahnen möglich ab einem $\eta$ von 0.1.\newline
Interresant zu beobachten ist, dass bei jeder Periode (2 Pi in den Einheiten des Plots)
um die Startposition herum die Genauigkeit stark zunimmt.
\begin{figure}
  \begin{center}
    \includegraphics[width=0.8\textwidth]{./images/2/energy-Heun.png}
  \end{center}
  \caption{Relative Energieänderung berechnet mit Heun}
  \label{fig:2body:heun:energy}
\end{figure}
\begin{figure}
  \begin{center}
    \includegraphics[width=0.8\textwidth]{./images/2/axis-Heun.png}
  \end{center}
  \caption{Relative \"Anderung der großen Halbachse berechnet mit Heun}
  \label{fig:2body:heun:axis}
\end{figure}
\begin{figure}
  \begin{center}
    \includegraphics[width=0.8\textwidth]{./images/2/excentric-Heun.png}
  \end{center}
  \caption{Relative \"Anderung der Exzentrizität berechnet mit Heun}
  \label{fig:2body:heun:excentric}
\end{figure}
\begin{figure}
  \begin{center}
    \includegraphics[width=0.8\textwidth]{./images/2/momentum-Heun.png}
  \end{center}
  \caption{Relative \"Anderung des Spezifischen Drehimpuls berechnet mit Heun}
  \label{fig:2body:heun:spec_angular_momentum}
\end{figure}
\subsubsection{Leapfrog/Verlet}
Um zu zeigen, dass Leapfrog/Verlet die Erhaltsungsgrößen erhält
werden mehrere Orbits (10) geplottet. Unter Abbildung \ref{fig:2body:verlet:energy}
sieht man deutlich, dass Verlet, selbst bei sehr großen Zeitschritten
die Energie erhält, auch über viele Perioden.
\begin{figure}
  \begin{center}
    \includegraphics[width=0.8\textwidth]{./images/2/energy-Verlet.png}
  \end{center}
  \caption{Relative Energieänderung berechnet mit Verlet}
  \label{fig:2body:verlet:energy}
\end{figure}
\begin{figure}
  \begin{center}
    \includegraphics[width=0.8\textwidth]{./images/2/axis-Verlet.png}
  \end{center}
  \caption{Relative \"Anderung der großen Halbachse berechnet mit Verlet}
  \label{fig:2body:verlet:axis}
\end{figure}
\begin{figure}
  \begin{center}
    \includegraphics[width=0.8\textwidth]{./images/2/excentric-Verlet.png}
  \end{center}
  \caption{Relative \"Anderung der Exzentrizität berechnet mit Verlet}
  \label{fig:2body:verlet:excentric}
\end{figure}
\begin{figure}
  \begin{center}
    \includegraphics[width=0.8\textwidth]{./images/2/momentum-Verlet.png}
  \end{center}
  \caption{Relative \"Anderung des Spezifischen Drehimpuls berechnet mit Verlet}
  \label{fig:2body:verlet:spec_angular_momentum}
\end{figure}
\subsubsection{RK4}
Das Verfahren Runge-Kutta 4ter Ordnung hat laut Theorie die beste Genauigkeit aller
benutzen Verfahren. Dies sieht man sofort in der Abbildung \ref{fig:2body:rk4:energy}
denn die Genauigkeit der Energie ist bis zur 16ten Potenz gegeben. Erneut verbessert
sich der Wert um die Anfangsposition herum. Bei RK4 sieht man jedoch noch eine weitere
interresante Tatsache: die endliche Genauigkeit der Fließkommazahlen im Rechner.
Bei sehr kleinen Zeitschritten z.B. $\eta = 0.0005$ sieht man, dass es zu Rundungsfehlern kommt.
Dabei kann man sehen, dass der Rechner doch die Zahlen diskret abspeichert und nicht kontinuierlich
rechnet.
\begin{figure}
  \begin{center}
    \includegraphics[width=0.8\textwidth]{./images/2/energy-RK4.png}
  \end{center}
  \caption{Relative Energieänderung berechnet mit RK4}
  \label{fig:2body:rk4:energy}
\end{figure}
\begin{figure}
  \begin{center}
    \includegraphics[width=0.8\textwidth]{./images/2/axis-RK4.png}
  \end{center}
  \caption{Relative \"Anderung der großen Halbachse berechnet mit RK4}
  \label{fig:2body:rk4:axis}
\end{figure}
\begin{figure}
  \begin{center}
    \includegraphics[width=0.8\textwidth]{./images/2/excentric-RK4.png}
  \end{center}
  \caption{Relative \"Anderung der Exzentrizität berechnet mit RK4}
  \label{fig:2body:rk4:excentric}
\end{figure}
\begin{figure}
  \begin{center}
    \includegraphics[width=0.8\textwidth]{./images/2/momentum-RK4.png}
  \end{center}
  \caption{Relative \"Anderung des Spezifischen Drehimpuls berechnet mit RK4}
  \label{fig:2body:rk4:spec_angular_momentum}
\end{figure}
\subsubsection{Hermit-Schema}
Betrachtet man das nicht iterierte Hermit Verfahren so fällt
auf dass die Genauigkeit der Energie etwa dem von Verlet entspricht, die Energie
jedoch nicht erhalten wird. Das Verlet-Verfahren ist dem einfachen Hermit-Schema vorzuziehen,
denn es erhält nicht nur die Energie langfristig, sondern schafft es zudem
ohne eine zusätzliche Berechnung der Ableitung der Beschleunigung.\newline\newline
\paragraph{Iteriertes Hermit Verfahren}
Benutzt man das iterierte Hermit Verfahren so fällt auf,
dass bei nur einer Iteration die Genauigkeit deutlich schlechter ist, als mit dem normalen Hermit Verfahren.
Nimmt man eine weitere Iteration hinzu, so merkt man schnell, dass das Verfahren schnell konvergiert.
Dabei gibt es kaum einen Unterschied zwischen 2 und 5 Iterationsschritten.


\section{Das N-Körper Problem [Aufgabe 3]}