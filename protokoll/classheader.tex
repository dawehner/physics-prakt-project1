%
% A. DOKUMENTKLASSE
% ---------------------------------------------------------------------------
%

%
%  2. Festlegen der Zeichencodierung des Dokuments und des Zeichensatzes.
%     Wir verwenden 'Latin1' (ISO-8859-1) f�r das Dokument,
%     und die 'T1' codierung f�r die Schrift.
%
\usepackage[utf8x]{inputenc}
\usepackage[T1]{fontenc}

%
%  3. Packet f�r die Index-Erstellung laden.
%
%\usepackage{makeidx}

%
%  4. Paket f�r die Lokalisierung ins Deutsche laden.
%     Wir verwenden neue deutsche Rechtschreibung mit 'ngerman'.
%
\usepackage[ngerman]{babel}


%
%  5. Paket f�r Anf�hrungszeichen laden.
%     Wir setzen den Stil auf 'swiss', und verwenden so die Schweizer Anf�hrungszeichen.
%
%\usepackage[style=swiss]{csquotes}


%
% 10. Paket f�r Farben an verschieden Stellen. 
%     Wir definieren zus�tzliche benannte Farben.
%
\usepackage{color}
\usepackage{framed}


%
% 11. Paket f�r spezielle PDF features.
%
\usepackage[%																								 Das Hyperref-Paket verwenden.
	pdftitle={Chaos},%			 Titel des PDF Dokuments.
	pdfauthor={Martin Schmidt, Daniel Wehner},%				     Autor des PDF Dokuments.
	pdfsubject={Protokoll zum Nbody},% 						 Thema des PDF Dokuments.
	pdfcreator={MiKTeX, LaTeX with hyperref and KOMA-Script},% Erzeuger des PDF Dokuments.
	pdfkeywords={Praktikum,Nbody,Compo-Praktikumg},
	bookmarksopenlevel=section,%															 Lesezeichen bis zu Sections �ffnen.
	pdfpagemode=UseOutlines,%                                  Inhaltsverzeichnis anzeigen beim �ffnen
	pdfdisplaydoctitle=true,%                                  Dokumenttitel statt Dateiname anzeigen.
	pdflang=de,%                                               Sprache des Dokuments.
	pdfstartview={FitH}%																			 Gr��e an Fensterbreite anpassen
]{hyperref}

%
% 12. Paket um Quellcode sauber zu formatieren.
%     Mit der option 'savemem' verschieben wir das laden von 
%     einzelnen Programmiersprachen auf einen sp�teren Zeitpunkt.
%
%\usepackage[savemem]{listings}

%
% 13a. Privates Paket f�r die Schriftart 'Goudy Sans' laden.
%      Dieses Paket ist nur f�r die publizierte Version des Dokuments gedacht
%      und an dieser Stelle mit den nachfolgenden Anweisungen auskommentiert.
%
%\usepackage{goudysans}

%
% 13a. Font 'Latin Modern Family' verwenden.
%      Verwende dieses Paket wenn du DML selbst kompilierst.
%
%\usepackage{lmodern}

%
% 14. Typewriter Font LuxiMono laden.
%



% 
% B. EINSTELLUNGEN
% ---------------------------------------------------------------------------
%

%
%  1. Definieren von eigenen benannten Farben.
%     F�r sp�tere Verwendung in dem Dokument, definieren wir einzelne
%     benannte Farben.
%
\definecolor{LinkColor}{rgb}{0,0,0.5}
\definecolor{HeadColor}{rgb}{0,0,0.6}
\definecolor{shadecolor}{rgb}{0.85,0.95,1}
%\definecolor{ListingBackground}{rgb}{0.85,0.85,0.85}

%
%  2. KOMA-Script Option, Zeilenumbruch bei Bildbeschreibungen.
%
\setcapindent{1em}

%
%  3. Stil der Kopf- und Fusszeilen.
%     Wir aktivieren mit 'headings' laufende Seitentitel.
%
\pagestyle{headings}


%
%  4. Stil der �berschriften auf normale Schrift.
%     Wir verwenden f�r die �berschriften den selben Font wie f�r den Text.
%
\setkomafont{chapter}{\huge\scshape\color{HeadColor}}       % Titel mit Normalschrift
\setkomafont{section}{\Large\color{HeadColor}}
\setkomafont{captionlabel}{\sffamily\bfseries\color{HeadColor}}     % Fette Beschriftungen 
\setkomafont{caption}{\sffamily}
\setkomafont{pagehead}{\sffamily\bfseries\color{HeadColor}}          % Kursive Seitentitel
\setkomafont{descriptionlabel}{\sffamily\bfseries} % Fette Beschreibungstitel
\setkomafont{pagenumber}{\sffamily\bfseries}
\setkomafont{part}{\Huge\scshape\color{HeadColor}}

%
%  5. Farbeinstellungen f�r die Links im PDF Dokument.
%
\hypersetup{%
	colorlinks=true,%        Aktivieren von farbigen Links im Dokument (keine Rahmen)
	linkcolor=LinkColor,%    Farbe festlegen.
	citecolor=LinkColor,%    Farbe festlegen.
	filecolor=linkColor,%    Farbe festlegen.
	menucolor=LinkColor,%    Farbe festlegen.
	urlcolor=LinkColor,%     Farbe von URL's im Dokument.
	bookmarksnumbered=true%  �berschriftsnummerierung im PDF Inhalt anzeigen.
}

%
% D. AKTIONEN
% ---------------------------------------------------------------------------
%

%
%  1. Index erzeugen.
%
\makeindex
%
%
% E. SILBENTRENNUNG
% ---------------------------------------------------------------------------
%

\hyphenation{De-zi-mal-trenn-zeichen In-stal-la-ti-ons-as-sis-tent}


%%%%%%%%%%%%%%%%%%%%%%%%%%%%%%%%%%%%%%%%%%%%%%%%%%%%%%%%%%%%
%
%  Adding lines above and below the chapter head
%
 
% 1st get a new command
\newcommand*{\ORIGchapterheadstartvskip}{}%
% 2nd save the original definition to the new command
\let\ORIGchapterheadstartvskip=\chapterheadstartvskip
% 3rd redefine the command using the saved original command
\renewcommand*{\chapterheadstartvskip}{%
  \ORIGchapterheadstartvskip
  {%
    \setlength{\parskip}{0pt}%
    \noindent\rule[.3\baselineskip]{\linewidth}{1pt}\par
  }%
}
 
% see above
\newcommand*{\ORIGchapterheadendvskip}{}%
\let\ORIGchapterheadendvskip=\chapterheadendvskip
\renewcommand*{\chapterheadendvskip}{%
  {%
    \setlength{\parskip}{0pt}%
    \noindent\rule[.3\baselineskip]{\linewidth}{1pt}\par
  }%
  \ORIGchapterheadendvskip
}
%
%  End of chapter head change
%
%%%%%%%%%%%%%%%%%%%%%%%%%%%%%%%%%%%%%%%%%%%%%%%%%%%%%%%%%%%%














%
% ===========================================================================
% EOF
%
