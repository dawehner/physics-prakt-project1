\section{Motivation}
Am Anfang des Physik-Studiums scheint es so, als ließen
sich alle Probleme auf analytisch lösbare Funktionen zurückführen,
doch bald merkt man das dies eher der Sonderfall ist.
Um trotzdem noch physikalische Systeme zu verstehen, kann
man entweder auf Experimente oder auf Simulationen zurückgreifen.
Dabei ist das N-Body Problem eins der klassischen Szenarien, welches sich auf
viele Bereiche anwenden l\"asst. Von einem einfachen 2-Körper System wie Sonne+Planet
wie es in diesem Versuch der Fall ist, bis zu dem kompletten Universum.
\footnote{\label{foot:2}http://www.mpa-garching.mpg.de/galform/presse/}
\newline
Im folgendem Versuch wird das N-Body Problem mithilfe eines in C++
geschrieben Programms gelöst und analysiert.
\section{Theorie}
