\section{Motivation}
Am Anfang des Physik-Studiums scheint es so, als ließen
sich alle Probleme auf analytisch lösbare Funktionen zurückführen,
doch bald merkt man das dies eher der Sonderfall ist.
Um trotzdem noch physikalische Systeme zu verstehen, kann
man entweder auf Experimente oder auf Simulationen zurückgreifen.
Dabei ist das N-Body Problem eins der klassischen Szenarien, welches sich auf
viele Bereiche anwenden l\"asst. Von einem einfachen 2-Körper System wie Sonne+Planet
wie es in diesem Versuch der Fall ist, bis zu dem kompletten Universum.
\footnote{\label{foot:2}http://www.mpa-garching.mpg.de/galform/presse/}
\newline
Im folgendem Versuch wird das N-Body Problem mithilfe eines in C++
geschrieben Programms gelöst und analysiert.
\section{Theorie}
\section{Programm}
Um das Programm aufzurufen verwende man verschiedene Parameter.
\lstset{language=Bash,
  basicstyle=\small \ttfamily,
  showspaces=false,
  showtabs=false,
  tab= ,
  keywordstyle=\bfseries,
  showstringspaces=false,
  framexleftmargin=0mm,
  frame=single,
  texcl=true}
\begin{lstlisting}
-i Integrationsmethode
  0 Euler
  1 Heun
  2 Verlet
  3 Leapfrog
  4 RK4
  5 Hermit
  8 Hermit 1 Iterationsschritt
  9 Hermit 2 Iterationsschritt
  10 Hermit 5 Iterationsschritt
-t Adaptiver Zeitschritt [ja/nein]
-o "Dateiname"
  Wohin soll die Ausgabe geschrieben werden
-f "Eingabe-Datei"
\end{lstlisting}
Mithilfe dieser Parameter wird die Eingabedatei gelesen, das System auf Gesamtmasse 1 normiert
und in das Schwerpunktsystem transferiert. Danach startet die Integration.\\
Dazu werden die erhaltenen Bahnpositionen und Geschwindigkeiten nach jedem Schritt in die Datei ``output.dat'' geschrieben
und die Erhaltungsgrößen wie Gesamtenergie, Gesamtimpuls und Gesamtdrehimpuls in die Datei ``output-conserved.dat''.
Speziell für das zwei Körper Problem werden zusätzlich noch in eine Datei ``output-conserved-2body.dat''
die Exzentrizität, der spezifische Drehimpuls und die große Halbachse für jeden Körper geschrieben.
Das Programm ist dafür ausgelegt diese 3 Werte auch für ein System mit einem schweren
Körper in der Mitte und zwei Satelliten zu berechnen.\\
